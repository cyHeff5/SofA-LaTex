\chapter{Implementierung}

Das Projekt ist in der Programmiersprache CSharp geschrieben.
Lokale Instanzen von Apache Kafka und Cassandra werden in Docker-Containern gestartet.
%TODO: Implementierung-Überblick


\section{Datenbankstruktur}


\section{Core-Bibliothek}
Auf der Core-Bibliothek bauen alle anderen Projekte auf.
In dieser Bibliothek befinden sich alle geteilten Programmteile:
\begin{itemize}
    \item \textbf{Konfiguration} (Kafka-Topics, Cassandra-Aufbau: Tabellen und -Spaltennamen)
    \item \textbf{Geteilte Interfaces} (ICarModel, ICarData, IParkingEventData)
    \item \textbf{Geteilte Datenstrukturen} (CarEntryData, CarExitData, OccupancyState)
    \item \textbf{Serialisierungs- und Deserialisierungsfuntionen} Die Datenstrukturen bin\"ar oder als String serialisieren und wiedererstellen.
    \item \textbf{CassandraHelper} Eine Hilfsklasse, die die Verbindung zur Cassandra-Datenbank verwaltet und alle in anderen Projekten verwendeten Datenbankzugriffe mit Hilfe von CQL (Cassandra Query Language) umsetzt.
\end{itemize}


\section{Datengrundlage}
Da keine realen Daten zur Verf\"ugung stehen, verwenden wir einen selbst geschriebenen Datengenerator, der Ein- und Ausfahrten in ein Parkhaus über Zeit simuliert.
Dabei werden Parkevent-Daten \quotes{CarEntryData} und \quotes{CarExitData} bestehend aus Parkhaus-Id, Zeitstempel, Nummernschild, Automarke, Automodell, Farbe, gemessener Länge, gemessener Breite, gemessener Höhe und gemessenem Gewicht.
Ausfahrts-Events werden stets basierend auf einem zuvor generierten Einfahrts-Event, zu dem es noch kein zugehöriges Ausfahrts-Event gibt generiert.
%Todo: Tabelle angeben
In der Datenbank stehen Einfahrts- und Ausfahrts-Events in der selben Tabelle und werden nur durch ein bool-Flag, bei dem \textbf{True} ein Einfahrts-Event markiert, unterschieden.


\section{Speed Layer}
Der Speed Layer beinhaltet nur ein einzelnes Projekt \quotes{SpeedLayerParkingDeckWorker}, das gleichzeitig Kafka-Consumer und Kafka-Producer ist.
Dabei handelt es sich um eine Kommandozeilenanwendung, die die eingehenden Grunddaten aus der Kafka-Topic \quotes{car-events} erhält und für ein einzelnes Parkhaus mehrere Events zu einer Zwischensumme zusammenfasst.
Diese Zwischensumme entspricht der Differenz der im Parkhaus stehenden Autos im Vergleich zum Stand vor den Events.
Wird eine Obergrenze an Park-Events für ein Parkhaus erreicht, oder ist eine Zeitspanne seit dem ersten Park-Event abgelaufen, so wird die Differenz gemeinsam mit der Parkhaus-Id in eine zweite Kafka-Topic \quotes{car-events-preprocessed} geschrieben.
Die Obergrenze an Parkevents, sowie die Zeitspanne sind im Worker konfigurierbar.
Es können beliebig viele Speed-Layer-Worker gleichzeitig gestartet werden, wobei je Parkhaus ein Speed-Layer-Worker vorgesehen ist.
Standardmäßig wird beim Starten des Projekts auch ein Worker je Parkhaus gestartet.
Durch die Verwendung der selben GruppenId (abhängig vom Parkhaus des Workers) ist es auch möglich, mehrere Worker für das selbe Parkhaus zu starten, während Kafka jedes Event nur an einen dieser Worker verteilt.
Praktisch ist das in unserer Fallstudie allerdings nicht sinnvoll, da ein einzelnes reales Parkhaus nur eine überschaubare Menge an Events produzieren kann.


\section{Batch Layer}
%TODO: Projektnamen in Repo anpassen
Das Batch Layer besteht aus mehreren Projekten:
\begin{itemize}
    \item \textbf{Kafka-Consumer} persistiert alle Grunddaten aus Kafka in der Datenbank.
    \item \textbf{Batch-Processing} führt regelmäßig verschiedene Batch-Berechnungen basierend auf den in der Datenbank persistierten Daten aus und schreibt die Ergebnisse in die Datenbank zurück. 
\end{itemize}

\subsection{KafkaConsumer}
Der Kafka-Consumer des Batch Layers ist eine Kommandozeilenanwendung, die beim Start die Kafka-Topic \quotes{car-events} abonniert und alle hier anfallenden Events erhält.
Diese Events werden dann in Cassandra übertragen, das geschieht der Einfachheit halber aktuell in einzelnen Datenbankoperationen.
In einem realen System, in dem mehr Daten anfallen, wäre hingegen wichtig, die Events zu bündeln und eine Menge Events gesammelt an die Datenbank zu übertragen.
Es können beliebig viele Kafka-Consumer gleichzeitig gestartet werden.
Alle Instanzen teilen die selbe Kafka-GroupId, wodurch jedes Event nur an genau eine Instanz übertragen wird.
In der Kommandozeile werden Ein- und Ausfahrten ausgegeben.

\subsection{Batch-Processing}
Das Batch-Processing des Batch Layers ist eine Kommandozeilenanwendung, die periodisch die Daten der vergangenen halben Stunde aus der Datenbank läd und darauf basierend Zwischenfüllstende je Parkhaus je 10 Minuten berechnet.
Da der Datengenerator beschleunigt Daten generiert, läuft diese Anwendung auch beschleunigt und berechnet alle 2 Minuten die Daten einer halben Stunde.



\section{Serving Layer}
Das Serving Layer besteht nur aus einem Projekt \quotes{ServingLayer}.
Hier handelt es sich um eine Kommandozeilenanwendung, die Daten aus Kafka und Cassandra zusammenführt und für Benutzeranwendungen bereitstellt.
Beim Start liest der Serving Layer alle existierenden Parkhäuser aus der Cassandra-Datenbank und legt sich je Parkhaus einen Cache an.
Dazu werden die Kafka-Topics \quotes{car-events-preprocessed}, deren Daten im Speed Layer berechnet werden, und \quotes{batch-notification} abonniert.
Dieser Cache beinhaltet die aktuelle Anzahl an Autos im Parkhaus, die sich aus dem Ergebnis der letzten Batch-Zwischensumme, sowie der seitdem angefallenen Summe an Park-Events in \quotes{car-events-preprocessed} ergibt.
Immer wenn eine Nachricht in \quotes{batch-notification} den Abschluss eines weiteren Batch-Vorgangs signalisiert, wird die aktuelle Batch-Zwischensumme neu geladen, sowie der veraltete Teil der Speed Layer Daten aus \quotes{car-events-preprocessed} verworfen.
Aufgrund des Aufbaus des Speed-Layers, sowie der asynchronen Natur von Batch- und Speed-Layer kann ein Ergebnis des Speed-Layers Daten aus 2 Batch-Berechnungen beinhalten.
Das kann zu temporären Inkonsistenzen führen, die aber nach der nächsten Batch-Berechnung korrigiert sind.

\section{Frontend}
Das Frontend ist in Avalonia 
%Avalonia
%nur mit 
