\chapter{Verwandte Arbeiten}
In diesem Kapitel geben wir einen Überblick über bestehende Arbeiten zur Lambda-Architektur und deren Einfluss auf die Entscheidungen, die wir bei der Entwicklung unserer Fallstudie getroffen haben. Zunächst erläutern wir die grundlegende Struktur der Lambda-Architektur, ihre Kernkomponenten und ihre Ziele. Anschließend betrachten wir existierende Implementierungen, die als Inspiration für unser eigenes Fallstudienszenario dienten.

Darauf aufbauend analysieren wir die architektonischen Eigenschaften der Lambda-Architektur sowie ihre bekannten Vor- und Nachteile, um eine Grundlage für die spätere Bewertung unserer Fallstudie zu schaffen. Abschließend vergleichen wir die Lambda-Architektur mit der Kappa-Architektur, einer alternativen Architektur, die in bestimmten Szenarien die Schwächen der Lambda-Architektur umgehen soll.

\section{Überblick über die Lambda-Architektur}
Die Lambda-Architektur wurde von Nathan Marz als Antwort auf die wachsenden Herausforderungen bei der Verarbeitung großer Datenmengen im Zeitalter von Big Data entwickelt. Sie entstand aus der Erkenntnis, dass herkömmliche Architekturen oft zu komplex, fehleranfällig und schwer skalierbar sind. In vielen Systemen kann bereits ein einziger Datenfehler oder der Ausfall einer Komponente dazu führen, dass das gesamte System inkonsistent wird, Daten verloren gehen oder verfälscht werden und das System nicht mehr zuverlässig arbeitet.

Ein weiteres Problem klassischer Architekturen ist, dass sie Echtzeit- und Batch-Verarbeitung nicht effizient kombinieren können. Bei Big-Data-Anwendungen muss das System nicht nur jederzeit auf die Daten zugreifen und auf Anfragen reagieren können, sondern auch sicherstellen, dass die gelieferten Informationen stets aktuell sind. Dies würde bedeuten, dass bei jeder einzelnen Anfrage alle gespeicherten Daten vollständig durchsucht und verarbeitet werden müssten, um die aktuellsten Werte zu gewährleisten. Bei wachsenden Datenmengen wird dies jedoch schnell ineffizient und führt zu hohen Latenzzeiten und Systembelastungen.

Um diese Probleme zu lösen, verfolgt die Lambda-Architektur einen Ansatz, der große Datenmengen effizient verarbeitet und gleichzeitig Fehlertoleranz, Skalierbarkeit und Konsistenz gewährleistet. Ein zentrales Konzept ist dabei die unveränderliche Speicherung von Daten (Immutable Data). Dadurch können historische Daten jederzeit erneut verarbeitet werden, ohne dass das System inkonsistent wird.

Das Hauptziel der Lambda-Architektur ist es, Echtzeit- und Batch-Verarbeitung strukturell zu trennen, um die Vorteile beider Methoden zu kombinieren. Zu diesem Zweck ist die Architektur in drei Schichten unterteilt (siehe Abbildung 1):

\begin{itemize}
	\item \textbf{Batch Layer:} Speichert alle Rohdaten unverändert und dient als Langzeitarchiv. Ein wesentlicher Aspekt dieser Schicht ist, dass die gespeicherten Daten nicht verändert werden, sondern ausschließlich für regelmäßige Langzeitanalysen verwendet werden.
	\item \textbf{Speed Layer:} Verarbeitet Datenströme in Echtzeit, um sofortige Ergebnisse zu liefern. Da der Batch-Layer aufgrund seiner Struktur nicht für schnelle Abfragen geeignet ist, übernimmt der Speed-Layer die Aufgabe, aktuelle Daten schnell zur Verfügung zu stellen.
	\item \textbf{Serving Layer:} Verbindet die Batch- und die Speed-Layer und stellt die Daten für Benutzeranfragen bereit. Kombiniert die Ergebnisse der beiden Schichten und ermöglicht so eine konsistente und effiziente Datenbereitstellung.
\end{itemize}

Durch diese klare Trennung der Verarbeitungsebenen ermöglicht die Lambda-Architektur ein Gleichgewicht zwischen geringer Latenz und hoher Genauigkeit. Während die Batch Layer eine robuste und fehlertolerante Verarbeitung großer Datenmengen gewährleistet, wäre es zu zeitaufwändig, diese Schicht bei jeder Anfrage vollständig zu durchlaufen, um aktuelle Daten bereitzustellen. Hier kommt der Speed Layer ins Spiel: Er verarbeitet neue Daten in Echtzeit und stellt sie sofort zur Verfügung, um eine schnelle Reaktionszeit zu gewährleisten.

\section{Anwendungsfälle der Lambda-Architektur}
Da traditionelle Systeme wie relationale Datenbanken und klassische Batch-Processing-Systeme zunehmend an ihre Grenzen stoßen, wenn es darum geht, die stetig wachsenden Datenmengen effizient zu verarbeiten, gewinnt die Lambda-Architektur für moderne Anwendungen immer mehr an Bedeutung. Insbesondere im Zeitalter des Internets, in dem kontinuierlich riesige Datenmengen generiert, übertragen und analysiert werden, hat sich die Lambda-Architektur als leistungsfähige Lösung etabliert.

Yuvraj Kumar (2020) beschreibt in seinem Paper eine Vielzahl von modernen Anwendungsfällen der Lambda-Architektur sowie die Technologien, die häufig für deren Implementierung verwendet werden. Große Technologieunternehmen wie Twitter, LinkedIn, Netflix und Amazon nutzen die Lambda-Architektur, um ihre enormen Datenmengen effizient zu analysieren. Vor allem im Bereich der personalisierten Werbung spielt sie eine zentrale Rolle: Historische Kundendaten werden mit aktuellen Nutzerinteraktionen kombiniert, um in Echtzeit maßgeschneiderte Werbung auszuspielen.

Auch im Finanzsektor nutzen Unternehmen die Lambda-Architektur, um historische Transaktionsdaten mit Echtzeitanalysen zu verknüpfen. So können verdächtige Muster identifiziert und Betrugsversuche frühzeitig erkannt werden. Auch im Internet of Things (IoT) findet die Lambda-Architektur breite Anwendung. So wird sie beispielsweise in Smart Cities eingesetzt, um Logistikprozesse zu optimieren oder die Abfallentsorgung effizienter zu gestalten.

Das Paper von Kiran et al. zeigt, dass insbesondere Smart City Anwendungen stark von der Lambda-Architektur profitieren, da hier große Mengen an Sensordaten kontinuierlich erfasst und analysiert werden müssen. Die Autoren beschreiben eine konkrete Implementierung der Lambda-Architektur zur Verarbeitung und Analyse von Sensordaten in einer Netzwerkumgebung.

Als praktischen Anwendungsfall untersuchen Kiran et al. die Verarbeitung von Netzwerk- und Sensordaten aus dem Energy Sciences Network (ESnet). Ziel der Implementierung ist die Erkennung von Anomalien und die Optimierung der Netzwerkauslastung. Die Lamda-Architektur ist für diese Anwendung besonders geeignet, da sie die gleichzeitige Verarbeitung historischer und aktueller Netzdaten ermöglicht. Während im Batch-Layer Langzeitanalysen durchgeführt werden, um wiederkehrende Muster zu identifizieren, kann der Speed-Layer Anomalien in Echtzeit erkennen und darauf reagieren. 

Die in der Literatur beschriebenen Anwendungsfälle, insbesondere die Arbeiten von Kiran et al. und Kumar, haben uns geholfen, ein geeignetes Szenario für unsere Fallstudie zu identifizieren. Die Forschung zeigt, dass die Lambda-Architektur besonders häufig in Smart City-Anwendungen eingesetzt wird, da sie die Echtzeitverarbeitung großer Sensordatenströme mit Langzeitanalysen kombiniert.

Darauf aufbauend wollten wir für unsere Fallstudie eine Anwendung im Smart City Kontext wählen, die ebenfalls Echtzeitanalysen mit Langzeitprognosen kombiniert. Dies führte uns schließlich zur Parkhausanalyse. Ähnlich wie in der Arbeit von Kiran et al. ist es in unserem Szenario sinnvoll, historische Daten zur Vorhersage der Parkhausauslastung mit Echtzeitinformationen über die aktuelle Belegung zu verknüpfen.
